%%%%%%%%%%%%%%%%%%%%%%%%%%%%%%%%%%%%%%%%%
% Presentación Beamer
% LaTeX Template
% Version 1.0 (10/11/12)
%
% This template has been downloaded from:
% http://www.LaTeXTemplates.com
%
% Plantilla para presentaciones de laboratorio de Comunicaciones 4
% Universidad de San Carlos de Guatemala
% Autor: Rodrigo Chang
%
% Licencia:
% CC BY-NC-SA 3.0 (http://creativecommons.org/licenses/by-nc-sa/3.0/)
%
%%%%%%%%%%%%%%%%%%%%%%%%%%%%%%%%%%%%%%%%%

%----------------------------------------------------------------------------------------
%	PAQUETES Y TEMAS
%----------------------------------------------------------------------------------------

\documentclass{beamer}

\mode<presentation> {
	
\usetheme{Madrid}
\usecolortheme{beaver}
%\usecolortheme{dolphin}
%\usecolortheme{orchid}
%\usecolortheme{wolverine}

% Configuracion de las diapositivas

% To remove the footer line in all slides uncomment this line
\setbeamertemplate{footline} 
% To replace the footer line in all slides with a simple slide count uncomment this line
%\setbeamertemplate{footline}[page number]
% To remove the navigation symbols from the bottom of all slides uncomment this line 
%\setbeamertemplate{navigation symbols}{}
}

\usepackage[utf8]{inputenc}
\usepackage[spanish, es-tabla]{babel}
\usepackage{graphicx} % Allows including images
\usepackage{booktabs} % Allows the use of \toprule, \midrule and \bottomrule in tables
\usepackage{pdfpages} % Para incluir paginas de otros PDFs
\usepackage{comment}
\usepackage{caption}

% Referencias
\usepackage[backend=biber, bibstyle=authoryear, citestyle=authoryear]{biblatex}
\addbibresource{referencias.bib}

%----------------------------------------------------------------------------------------
%	CONFIGURACION DE LISTINGS
%----------------------------------------------------------------------------------------
% Definiciones para el estilo de python
% Paquetes necesarios
\usepackage{listingsutf8}
\usepackage{color}

% Redefinición de titulos de caption
\renewcommand\lstlistingname{Código}
\renewcommand\lstlistlistingname{Índice de códigos}

% Definición de colores para R
\definecolor{frame_numbers_color}{rgb}{0.5,0.5,0.5}

\definecolor{r_comment_color}{RGB}{221, 0, 221}
\definecolor{r_string_color}{RGB}{0, 170, 0}
\definecolor{r_keyword_color}{RGB}{255, 119, 0}
\definecolor{r_identifier_color}{RGB}{0,0,0}

% Definicion de estilo para R
\lstdefinestyle{customR}{
	backgroundcolor=\color{white},   % choose the background color;
	%basicstyle=\footnotesize,        % the size of the fonts that are used for the code
	basicstyle=\small\ttfamily,
	breakatwhitespace=false,         % sets if automatic breaks should only happen at whitespace
	breaklines=true,                 % sets automatic line breaking
	captionpos=t,                    % sets the caption-position to bottom
	commentstyle=\color{r_comment_color},    % comment style
	deletekeywords={...},            % if you want to delete keywords from the given language
	escapeinside={\%*}{*)},          % if you want to add LaTeX within your code
	extendedchars=true,              % lets you use non-ASCII characters; for 8-bits encodings only, does not work with UTF-8
	frame=single,                    % adds a frame around the code
	framesep=3pt,
	identifierstyle=\color{r_identifier_color},
	keepspaces=true,                 % keeps spaces in text, useful for keeping indentation of code (possibly needs columns=flexible)
	keywordstyle=\mdseries\color{r_keyword_color},       % keyword style
	language=R,                 % the language of the code
	morekeywords={},            % if you want to add more keywords to the set
	numbers=left,                    % where to put the line-numbers; possible values are (none, left, right)
	numbersep=7pt,                   % how far the line-numbers are from the code
	numberstyle=\tiny\color{frame_numbers_color}, % the style that is used for the line-numbers
	rulecolor=\color{black},         % if not set, the frame-color may be changed on line-breaks within not-black text (e.g. comments (green here))
	showspaces=false,                % show spaces everywhere adding particular underscores; it overrides 'showstringspaces'
	showstringspaces=false,          % underline spaces within strings only
	showtabs=false,                  % show tabs within strings adding particular underscores
	stepnumber=1,                    % the step between two line-numbers. If it's 1, each line will be numbered
	stringstyle=\color{r_string_color},     % string literal style
	tabsize=4,                       % sets default tabsize to 2 spaces
	title=\lstname                   % show the filename of files included with \lstinputlisting; also try caption instead of title	
}
\lstset{style=customR, frame=none}

%\begin{frame}[fragile]{Configuración con registros}
%	\begin{lstlisting}
%	
%	\end{lstlisting}
%\end{frame}


%----------------------------------------------------------------------------------------
%	CONFIGURACION DE SLIDES DE AYUDA
%----------------------------------------------------------------------------------------
\newcommand{\incluirpdf}[3]{
	\setbeamercolor{background canvas}{bg=}
	\includepdf[pages=#2-#3]{#1}}

\newcommand{\figura}[2]{
	\begin{figure}
		\centering
		\includegraphics[width=#1\textwidth]{#2}
	\end{figure}	
}

\newcommand{\diapositivatexto}[2]{
\begin{frame}{#1}
	\huge\centerline{#2}
\end{frame}
}
%----------------------------------------------------------------------------------------
%	PÁGINA DEL TITULO
%----------------------------------------------------------------------------------------

 % El titulo corto aparece al final de cada diapositiva, 
 % el titulo completo sólo en la página del titulo
%\title[Laboratorio de Comunicaciones 4]{Módulo ADC y muestreo}
\title[Laboratorio de microcontroladores]{Modelo de pronóstico del tipo de cambio para Guatemala utilizando redes neuronales artificiales}

% Nombre del autor
\author{Rodrigo Chang}
% Institución al fondo de cada diapositiva, abreviación
\institute[USAC]
{
% Nombre de la institución para la página del titulo
Banco de Guatemala \\ 
\medskip
% Dirección de correo electrónico
\textit{Programa de Estudios Superiores} % Dirección de correo electrónico
}
\date{julio de 2016} % Fecha, que puede ser cambiada a una fecha específica

%-------------------------------------------------------
%	INICIO DEL DOCUMENTO
%-------------------------------------------------------

\newcommand{\anchoGraficas}{0.9}

\begin{document}

\begin{frame}
% Imprime la diapositiva del titulo como primera página
\titlepage 
\end{frame}

\begin{comment}
% Para hacer una diapositiva como índice
\begin{frame}
% Table of contents slide, comment this block out to remove it
\frametitle{Lo que veremos hoy...} 
\tableofcontents
\end{frame}
\end{comment}

%--------------------------------------------------------
%	MARCO TEÓRICO
%--------------------------------------------------------
\section{Marco teórico}

\subsection{El tipo de cambio}

\begin{frame}{El tipo de cambio}
	\begin{itemize}
		\item El precio de la moneda de un país en función de la moneda de otro.
		\item Fuerte impacto sobre la cuenta corriente y otras variables macroeconómicas.
		\item Permite comparar los precios de bienes y servicios.
	\end{itemize}
\end{frame}

\begin{comment}
\begin{frame}{El mercado de divisas}
	Tres sectores de comercio:
	\begin{itemize}
		\item El comercio interbancario, que conforma entre un 60\% a 80\% del mercado \parencite{exchecon}.
		
		\item El comercio manejado a través de corredores de instituciones financieras no bancarias, que conforma de un 15\% a un 35\%.
		
		\item Y el comercio manejado por clientes privados y empresas multinacionales, que conforma alrededor del 5\%.
	\end{itemize}
	
	Además: 
	
	\begin{itemize}
		\item Intervención de bancos centrales en el mercado 
		%: participantes del mercado observan los movimientos de los bancos centrales para obtener información sobre la futura política macroeconómica.
	\end{itemize}
\end{frame}
\end{comment}

\subsection{El modelo monetario}
\begin{frame}{El modelo monetario de precios rígidos}
	De acuerdo con \textcite{boughton1988monetary}, el enfoque monetario comprende las siguientes cinco hipótesis: 
	\begin{enumerate}
		\item La PPP se cumple sobre un horizonte de tiempo relevante.
		\item Se mantiene la paridad de interés descubierta en todo momento.
		\item La demanda de saldos reales de dinero es una función estable de un conjunto reducido de variables.
		\item La oferta de dinero se determina por un proceso estable.
		\item Las expectativas son en algún sentido racionales.
	\end{enumerate}
	
	\begin{equation}
	s = (m - m^*) - \phi (y-y^*) + \lambda (\pi - \pi^*) - (1/\theta)[(i-\pi) - (i^* - \pi^*)]
	\label{spma}
	\end{equation}
\end{frame}

\subsection{Crítica de Meese y Roggoff}

\begin{frame}{Crítica de \textcite{meese1983empirical}}
	\begin{itemize}
		\item Fundamentos macroeconómicos vs. caminata aleatoria.
		
		\item Implicaciones prácticas: permite a inversores, bancos centrales e instituciones financieras decidir la asignación de activos, manejar el riesgo y formular políticas.
		
		\item Implicaciones teóricas: la significancia teórica del pronóstico del tipo de cambio tiene implicaciones vitales en la hipótesis de mercados eficientes, así como el desarrollo de modelos teóricos en finanzas internacionales.
	\end{itemize}
\end{frame}

%--------------------------------------------------------
%	METODOLOGÍA
%--------------------------------------------------------
\section{Metodología}

\begin{frame}{Metodología}{Objetivo}
	Se utilizan 3 modelos distintos para comparar sus resultados de pronóstico:
	\begin{itemize}
		\item Modelo de caminata aleatoria
		\item Modelo lineal
		\item Modelo con redes neuronales artificiales
	\end{itemize}
\end{frame}

\subsection{Caminata aleatoria}

\begin{frame}{Metodología}{Caminata aleatoria}
	Siguiendo la metodología de \textcite{meese1983empirical}, se propone un modelo de caminata aleatoria de la forma: 
	
	\begin{equation}
	s_t = s_{t-1} + \epsilon_t
	\end{equation}
	
	donde:
	\begin{itemize}
		\item $\epsilon_t \sim \mathrm{Normal}(0, \sigma_e^2)$, donde $\sigma_e^2$ es la varianza de las observaciones dentro de la muestra
	\end{itemize}
\end{frame}

\subsection{Modelo lineal}

\begin{frame}{Metodología}{Modelo lineal}
	La especificación exacta del modelo lineal utilizado es la siguiente:
	
	\begin{equation}
	s_t = \beta_0 + \beta_1 X_{1t} + \beta_2 X_{2t} + \beta_3 X_{3t} + \beta_4 X_{4t} + \epsilon_t
	\label{linearModel}
	\end{equation}
	
	donde:
	
	\begin{itemize}
		\item $X_1$ es el logaritmo natural de la oferta relativa de dinero (como índices): $ \ln \big ( \frac{M2_{GT}}{M2_{US}} \big) $
		
		\item $X_2$ es el logaritmo natural del PIB relativo (como índices): $ \ln \big ( \frac{PIB_{GT}}{PIB_{US}} \big) $
		
		\item $X_3$ es el diferencial de tasas de interés: $ ( i_{GT} - i_{US} ) $
		
		\item $X_4$ es el diferencial de la inflación esperada de largo plazo $ ( \pi_{GT} - \pi_{US} ) $
		
	\end{itemize}
\end{frame}

\subsection{Redes neuronales}
% Acá va qué son las redes neuronales y como se estiman, y la especificación final

\begin{frame}{Metodología}{Modelos de redes neuronales artificiales}
	\figura{0.6}{figuras/neuralNetwork.pdf}
	
	\begin{equation}
	\hat{Y} = f \bigg ( w_0 + \sum_{j=1}^{J} \phi \big ( w_{0j} + \mathbf{w_j}^T \mathbf{x} \big )\bigg )
	\label{ann_model}
	\end{equation}
\end{frame}

\begin{frame}{Metodología}{Gradiente en descenso}
	\figura{1}{figuras/BackPropAlg.png}
\end{frame}


\begin{frame}{Metodología}{Especificación del modelo de RNAs}
	La especificación exacta del modelo utilizado es la siguiente:
	
	\begin{equation}
	s_t = f(X_{1t}, X_{2t}, X_{3t}, X_{4t}) + \epsilon_t
	\label{annModel}
	\end{equation}
	
	donde $f$ representa la aproximación de la red neuronal a los datos.\\
	
\end{frame}

\begin{frame}{Metodología}{Arquitectura de RNA}
	\figura{0.7}{figuras/spec_ann.pdf}
\end{frame}


%--------------------------------------------------------
%	RESULTADOS
%--------------------------------------------------------
\section{Resultados}

\subsection{Caminata aleatoria}

\begin{frame}{Resultados}{Caminata aleatoria fuera de la muestra}
	\figura{\anchoGraficas}{figuras/RW_out.png}
\end{frame}


\subsection{Modelo lineal}

\begin{frame}{Resultados}{Modelo lineal dentro de la muestra}
	\figura{\anchoGraficas}{figuras/lin_in.png}
\end{frame}


\begin{frame}{Resultados}{Modelo lineal fuera de la muestra}
	\figura{\anchoGraficas}{figuras/lin_out.png}
\end{frame}

\subsection{Redes neuronales}

\begin{frame}{Resultados}{Modelo de redes neuronales dentro de la muestra}
	\figura{\anchoGraficas}{figuras/ann19_in.png}
\end{frame}

\begin{frame}{Resultados}{Modelo de redes neuronales fuera de la muestra}
	\figura{\anchoGraficas}{figuras/ann19_out.png}
\end{frame}

\subsection{Resumen de resultados}

\begin{frame}{Resultados}{Resumen de resultados}
	% Table generated by Excel2LaTeX from sheet 'Tablas'
\begin{table}[htbp]
  \centering
  \caption{Resultados de pronóstico del tipo de cambio Q/USD}
    \begin{tabular}{lrrr}
    \toprule
    \multicolumn{1}{c}{\textbf{Fuera de la muestra}} & \multicolumn{1}{c}{\textbf{RMSE}} & \multicolumn{1}{c}{\textbf{RMSPE}} & \multicolumn{1}{c}{\textbf{PERC}} \\
    \midrule
    Caminata aleatoria    & 0.02763 & 0.01347 & 31\% \\
    Modelo lineal & 0.02512 & 0.01234 & 43\% \\
    RNA con 8 neuronas & 0.02059 & 0.01011 & 57\% \\
    RNA con 9 neuronas & 0.01969 & 0.00966 & 46\% \\
    \bottomrule
    \end{tabular}%
    \caption*{Fuente: elaboración propia.}
  \label{tab:resOutOfSample}%
\end{table}%

\end{frame}

\begin{frame}{Resultados}{Relación de error entre modelos}
	% Table generated by Excel2LaTeX from sheet 'Tablas'

\begin{table}[htbp]
	\centering
	\caption{Relación de error entre modelos de pronóstico}
	\begin{tabular}{lrr}
		\toprule
		\textbf{Modelo} & \textbf{RMSE/RW} & \textbf{RMSPE/RW} \\
		\midrule
		Lineal & 0.9090 & 0.9156 \\
		%RNA\_18 & 0.7452 & 0.7502 \\
		Red neuronal & 0.7127 & 0.7171 \\
		\bottomrule
	\end{tabular}%
	\label{tab:ratioError}%
	\caption*{Fuente: elaboración propia.}
\end{table}

\end{frame}

\begin{frame}{Resultados}{Importancia relativa entre variables}
	\figura{0.65}{figuras/garson_19.pdf}
\end{frame}

\begin{frame}{Metodología}{Diagrama de identificación neuronal de RNA}
\figura{1}{figuras/plotnet_19.pdf}
\end{frame}

\begin{frame}{Resultados}{Prueba de Diebold y Mariano}
	{\scriptsize
	% Table generated by Excel2LaTeX from sheet 'Hoja1'
\renewcommand{\baselinestretch}{1.5}
\begin{table}[htb]
  \centering
  \caption{Comparación entre modelos de pronóstico con horizonte de 10 meses}
    \begin{tabular}{cccccc}
    \toprule
    \multicolumn{1}{l}{Modelo 1} & Camin. Aleat. & Camin. Aleat. & RNA9  & Lineal & RNA9 \\
    \multicolumn{1}{l}{Modelo 2} & RNA9  & Lineal & Camin. Aleat. & Camin. Aleat. & Lineal \\
    \midrule
    Horizonte & \multicolumn{5}{c}{Valor p} \\
    \midrule
    \multicolumn{1}{c}{1} & 0.9492 & 0.1491 & 0.0508 & 0.8509 & 0.0272 \\
    \multicolumn{1}{c}{2} & 0.9312 & 0.1832 & 0.0688 & 0.8168 & 0.0260 \\
    \multicolumn{1}{c}{3} & 0.9386 & 0.1883 & 0.0614 & 0.8117 & 0.0304 \\
    \multicolumn{1}{c}{4} & 0.9447 & 0.1969 & 0.0553 & 0.8031 & 0.0471 \\
    \multicolumn{1}{c}{5} & 0.9166 & 0.2284 & 0.0834 & 0.7716 & 0.0506 \\
    \multicolumn{1}{c}{6} & 0.8694 & 0.2689 & 0.1306 & 0.7311 & 0.0518 \\
    \multicolumn{1}{c}{7} & 0.8704 & 0.2744 & 0.1296 & 0.7256 & 0.0525 \\
    \multicolumn{1}{c}{8} & 0.8760 & 0.2785 & 0.1240 & 0.7215 & 0.0778 \\
    \multicolumn{1}{c}{9} & 0.8757 & 0.2863 & 0.1243 & 0.7137 & 0.1112 \\
    \multicolumn{1}{c}{10} & 0.8807 & 0.2870 & 0.1193 & 0.7130 & 0.1058 \\
    \bottomrule
    \end{tabular}%
    \caption*{Fuente: elaboración propia.}
  \label{tab:dmtest}%
\end{table}%
\renewcommand{\baselinestretch}{1.5}
}
\end{frame}


%--------------------------------------------------------
%	CONCLUSIONES
%--------------------------------------------------------
\section{Conclusiones}

\begin{frame}{Observaciones finales}
	\begin{itemize}
		\item Redes neuronales artificiales como método de estimación para modelos econométricos.
		
		\item Pronóstico del tipo de cambio en Guatemala, corto y mediano plazo.
		
		\item Importancia de las variables: oferta relativa de dinero y tasas de interés.
	\end{itemize}
\end{frame}

%--------------------------------------------------------
%	DIAPOSITIVA FINAL Y FIN DEL DOCUMENTO
%--------------------------------------------------------
\begin{comment}
\begin{frame}{¡Gracias por su atención!}{Sección de dudas}
\figura{0.5}{figuras/philosoraptor.jpg}
\end{frame}
\end{comment}

\diapositivatexto{Sección de dudas y comentarios}{¡Gracias por su atención!}


\end{document} 