\section{Observaciones finales}

Las principales observaciones de este trabajo son: 

\begin{itemize}
	%\item Las redes neuronales artificiales conforman un método no lineal de estimación de modelos econométricos que permite realizar pronósticos, ya que pueden aprender el patrón de movimiento de las variables, y en cierta medida sirven para comprobar la relación existente entre las variables explicativas y la variable respuesta del modelo.
	
	\item Las redes neuronales artificiales conforman un método no lineal de estimación de modelos econométricos que permitieron realizar pronósticos, ya que pueden aprender el patrón de movimiento de las variables, y en cierta medida sirven para comprobar la relación existente entre el tipo de cambio y otras variables explicativas del modelo monetario. Finalmente, el modelo de redes neuronales artificiales en la predicción del tipo de cambio mostró los mejores resultados tanto en magnitud como dirección de los pronósticos, en comparación con un modelo monetario lineal, y con uno de caminata aleatoria.
	
	\item Se pudo pronosticar el tipo de cambio en Guatemala en un corto y mediano plazo a partir de los fundamentos macroeconómicos de un modelo monetario, resultado que va en contra del paradigma actual de la literatura, en el cual no es posible obtener mejores pronósticos que con un modelo de caminata aleatoria.
	
	\item En cuanto a la importancia de las variables explicativas, el modelo monetario lineal y el de redes neuronales resaltan la importancia del diferencial de tasas de interés entre Guatemala y Estados Unidos, así como la importancia de la oferta monetaria relativa mostrada por el modelo de redes neuronales, lo que indica que el tipo de cambio está principalmente explicado por las variables  guiadas a través de la política monetaria de Guatemala y Estados Unidos.
\end{itemize}