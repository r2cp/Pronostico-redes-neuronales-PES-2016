	\section{Marco teórico}
	
	% definición
	% importancia del estudio del tipo de cambio en otras variables
	% modelos o teorías disponibles que lo han estudiado
	% evolución histórica mundial del tipo de cambio y luego local
	
	\subsection{El tipo de cambio} 
	El tipo de cambio se puede definir como el precio de la moneda de un país en función de la moneda de otro. Debido a su fuerte impacto sobre la cuenta corriente y otras variables macroeconómicas, el tipo de cambio es uno de los precios más importantes de una economía abierta. El papel del tipo de cambio en el comercio internacional es fundamental, porque permite comparar los precios de bienes y servicios producidos en diferentes países. \parencite{intecon}. La anterior definición implica que un incremento en el tipo de cambio representa una depreciación de la moneda local, y un decremento en el tipo de cambio representa una apreciación de la moneda local. \parencite{exchecon}\\
	
	% El tipo de cambio se expresa como el número de unidades de moneda nacional que hay que entregar para obtener una unidad de moneda extranjera. 
	
	% Mercado de divisas
	Las transacciones entre monedas de diferentes países se llevan a cabo en el mercado de divisas, y son las que determinan el precio o tipo de cambio de una moneda respecto a otra. Una divisa es toda moneda extranjera usada en un país o región diferente del país donde se expide. \parencite{mochon}\\
	
	En el mercado de divisas se distinguen dos segmentos en función del tiempo entre el inicio y el cierre de las transacciones, que configuran dos grupos específicos de operaciones y dos precios o tipos de cambio: el tipo de cambio \textit{spot} y el tipo de cambio a plazo (\textit{forward}). En el mercado \textit{spot} se realizan operaciones de compra y venta de divisas, y la entrega se realiza dos días hábiles posteriores al de contratación, mientras que en el mercado a plazo la liquidación se realiza en el futuro, a partir del tercer día hábil posterior al de la contratación \parencite{mochon}.\\
	
	El mercado de divisas difiere de otros mercados financieros por tener un rol en tres sectores de comercio:
	\begin{itemize}
		\item El comercio interbancario, que conforma entre un 60\% a 80\% del mercado \parencite{exchecon}. La mayor parte de las transacciones se realizan a través de intercambio de depósitos bancarios denominados en diferentes moneda \parencite{intecon}.
		
		\item El comercio manejado a través de corredores de instituciones financieras no bancarias, que conforma de un 15\% a un 35\%, debido a la liberalización de los mercados financieros y el aumento de la diversificación de servicios ofrecidos por instituciones no financieras. \parencite{exchecon,intecon}
		
		\item Y el comercio manejado por clientes privados y empresas multinacionales (que conforma alrededor del 5\%), que se utiliza para el pagos de nóminas a empleados en otros países y para recibir ingresos en monedas diferentes de los países en que se establecen. \parencite{exchecon,intecon}
	\end{itemize}
	
	Además, los bancos centrales intervienen en el mercado de divisas, y su impacto en este puede ser importante, debido a que los participantes del mercado observan los movimientos de los bancos centrales para obtener información sobre la futura política macroeconómica, que puede afectar el tipo de cambio. \parencite{intecon}.
		
	% Tipos de cambio flotantes, fijos, etc	
	\subsubsection{Régimen de tipo de cambio}
	Un régimen cambiario es el conjunto de reglas que rigen la forma en que se determina el tipo de cambio \parencite{banguatregime}. La literatura económica distingue usualmente entre dos regímenes extremos: el régimen cambiario fijo, y el flotante o flexible  \parencite{exchecon}. \textcite{mochon} distinguen también los sistemas mixtos, o semifijos o ajustables, y la vigencia de un sistema de tipo de cambio u otro depende del grado de intervención del banco central.\\
	
	En los sistemas de tipo de cambio flexible, el tipo de cambio se ajusta de acuerdo al comportamiento de la ley de oferta y demanda de divisas. Una de las limitaciones de un régimen flexible es que pueden surgir problemas en las exportaciones e importaciones por la sensibilidad a las variaciones del tipo de cambio \parencite{mochon}.\\
	
	Los bancos centrales pueden intervenir en los mercados de divisas e incidir en la evolución del tipo de cambio a través de intervenciones de corto plazo para alterar el mismo en una determinada dirección. A este sistema, que funciona esencialmente como un régimen flexible, pero en el cual existe intervención, se le conoce sistema de flotación sucia \parencite{mochon}.\\

	Bajo un régimen de tipo de cambio fijo, el banco central fija el valor de su moneda respecto a otra, e interviene en el mercado de divisas con el objetivo de mantener el tipo de cambio en el nivel fijado, comprando o vendiendo divisas \parencite{mochon}.\\
	
	El Fondo Monetario Internacional clasifica los regímenes cambiarios de los miembros basado en su grado de flexibilidad y la existencia de compromisos formales o informales en la trayectoria del tipo de cambio, para ayudar a evaluar las implicaciones de la elección del régimen cambiarlo para el grado de independencia de la política monetaria. \parencite{imfregime}
	
	% Tipo de cambio como activo financiero
	\subsubsection{Tipo de cambio como activo financiero}
	El tipo de cambio es también el precio de un activo financiero, y los principios aplicables al comportamiento de los precios de activos puede considerarse igualmente para estudiar el comportamiento del tipo de cambio. \parencite{intecon}.\\
	
	La popularidad de este punto de vista puede atribuirse al realismo convincente en el mundo actual, tanto por su supuesto teórico distintivo, como por su aplicación empírica. La asunción teórica que comparten todos los activos de mercado es la ausencia de costos de transacción sustanciales, control de capitales, u otros impedimentos al flujo de capitales entre países. A esta asunción se le conoce como movilidad perfecta de capitales. La implicación empírica es que el tipo de cambio bajo un régimen flotante exhibe gran variabilidad, que excede a la que se puede atribuir a los determinantes macroeconómicos subyacentes \parencite{frankel1993exchange}\\
	
	La interpretación del tipo de cambio como precios de activos da como resultado una intuición importante de por qué es más volátil que las variables macroeconómicas subyacentes. \parencite{exchecon}
	
	
	\subsection{Paridad del poder adquisitivo}
	De acuerdo con \textcite[311]{mochon}, el tipo de cambio real ``es la relación a la que se pueden intercambiar los bienes y servicios de un país por los de otro''. El tipo de cambio real mide la relación entre el precio de una canasta de bienes y servicios de un país en relación con los precios del extranjero a través del tipo de cambio, y puede expresarse en forma logarítmica como $e$:
	
	\begin{equation}
		e = s + p - p^*
	\end{equation}
	
	Donde $s$ representa el logaritmo tipo de cambio nominal, $p$ el logaritmo del nivel de precios doméstico, y $p^*$ el logaritmo del nivel de precios extranjero.\\
	
	Por otro lado, La paridad del poder adquisitivo (PPP, por sus siglas en inglés) es una teoría que sostiene que el tipo de cambio nominal entre dos países debe ser igual a la razón entre el nivel de precios entre dos países, de manera que la unidad monetaria de un país tenga el mismo poder de compra, o adquisitivo, en otro país \parencite{banguatppp}. De forma logarítmica puede expresarse como\footnote{\textcite{frankel1993exchange} utiliza esta definición cuando la PPP se cumple de forma continúa. En el caso del modelo monetario de precios rígidos, se utiliza una versión de largo plazo descrita como $\bar{s} = \bar{p} - \bar{p}^* $ }: 
	
	\begin{equation}
		s = p - p^* 
		\label{ppp}
	\end{equation}
	
	Donde $s$ es el logaritmo del tipo de cambio nominal, $p$ el logaritmo del nivel de precios doméstico, y $p^*$ el nivel de precios extranjero.\\
	
	Este concepto ha sido utilizado ampliamente para medir los valores de equilibro de las monedas entre países, y es al concepto al que recurre un economista en primer lugar cuando se pregunta si una moneda está sobrevalorada o no. Asimismo, es una idea que guarda una relación estrecha con los modelos de tipo de cambio \parencite{exchecon}.\\
	
	De acuerdo con \textcite{boughton1988monetary}, una proposición relativamente débil acerca de la PPP es que el tipo de cambio nominal entre dos monedas se moverá en línea con el diferencial de inflación esperada entre los dos países. Asimismo, una proposición más fuerte es que el tipo de cambio real tenderá un nivel de equilibrio invariante en el tiempo, determinado de alguna manera por la ley del único precio, que establece que un mismo bien no puede venderse simultáneamente a diferentes precios en lugares distintos \parencite{mochon}.
	
	\subsection{Paridad del tipo de interés}
	Cuando no se asume riesgo de impago o control de capitales futuros, la movilidad perfecta de capitales implica la paridad de interés cubierta: la tasa de interés de un bono doméstico es igual a la tasa de interés en un bono extranjero más una prima futura sobre el tipo de cambio \parencite{frankel1993exchange}.\\
	
	La hipótesis de la paridad descubierta de interés establece que los retornos esperados sobre activos de renta fija serán iguales, sin importar la moneda de denominación \parencite{boughton1988monetary}. De acuerdo con \textcite{frankel1993exchange}: la tasa de interés de un bono doméstico es igual a la tasa de interés de un bono extranjero más la tasa esperada de apreciación de la moneda extranjera, lo que puede expresarse como: 
	
	\begin{equation}
	i = i^* + E(\Delta s)
	\label{uip}
	\end{equation}
	
	Donde $i$ representa el interés doméstico, $i^*$ el interés extranjero, y $E(\Delta s)$ la tasa de apreciación esperada sobre el tipo de cambio nominal $s$.
	
	% rol de las expectativas
	
	%\subsection{Importancia del estudio del tipo de cambio}
	
	\subsection{Modelos teóricos}
	
	Utilizando la perspectiva del tipo de cambio como el precio de un activo existen dos clases de modelos que describen el tipo de cambio: el modelo monetario, y el modelo de cartera de balance \parencite{exchecon}. Dicha dicotomía diferencia los modelos de acuerdo a si los bonos domésticos y bonos extranjeros se asumen como perfectos sustitutos en los portafolios de los tenedores de activos \parencite{frankel1993exchange}.\\
	% La diferencia subyace en que en el modelo monetario los activos que no son dinero (bonos) se asumen como perfectos sustitutos, mientras que en el modelo de cartera de balance se asumen sustitutos imperfectos  \parencite{exchecon}.
	
	%Esta sustitución consiste en que los tenedores de activos son indiferentes a la composición de su portafolio de bonos mientras la tasa esperada de retorno de los bonos de ambos países es la misma cuando se expresa en cualquier numerario común. Esto implica la paridad descubierta del tipo de interés \parencite{frankel1993exchange}.\\
	
	La sustituibilidad entre bonos domésticos y extranjeros y el supuesto de que los tenedores de activos sean indiferentes a la composición de su portafolio de bonos mientras la tasa de retorno esperada de los bonos de ambos países sea la misma cuando se expresa en un numerario común. Esto implica la paridad de interés descubierta \parencite{frankel1993exchange}.\\ 
	
	Para modelar el comportamiento del tipo de cambio nominal utilizando una aproximación basada en fundamentos macroeconómicos se ha utilizado ampliamente el modelo monetario, que se ha vuelto un caballo de batalla en la literatura de tipo de cambio. Este modelo se compone de dos variantes: una variante flexible y una con precios rígidos \parencite{exchecon}.\\
	
	De acuerdo con \textcite{boughton1988monetary}, el enfoque monetario comprende las siguientes cinco hipótesis: 
	\begin{enumerate}
		\item La PPP se cumple sobre un horizonte de tiempo relevante.
		\item Se mantiene la paridad de interés descubierta en todo momento.
		\item La demanda de saldos reales de dinero es una función estable de un conjunto reducido de variables.
		\item La oferta de dinero se determina por un proceso estable.
		\item Las expectativas son en algún sentido racionales.
	\end{enumerate}
	
	\subsubsection{El modelo monetario de precios flexibles}
	
	El enfoque monetario asume que no hay barreras que segmenten los mercados de capitales internacionales, como costos de transacción y control de capitales, y además, que los bonos domésticos y extranjeros son sustitutos perfectos en las funciones de demanda de los consumidores, es decir, solo hay un bono en el mundo \parencite{frankel1993exchange}.\\
	
	El modelo monetario flexible se basa en el supuesto de que en el mercado de bienes hay un solo bien en el mundo, lo que implica que existe paridad de poder de adquisitivo, y que esta se cumple de forma continua. \parencite{frankel1993exchange, exchecon}.\\
	
	% Acerca del fallo de PPP - esto talvez va en PPP
	Empíricamente se han observado muchas fallas de corto plazo en el cumplimiento de la paridad de poder adquisitivo \parencite{frankel1993exchange}. Durante los últimos diez años la investigación en paridad de poder adquisitivo ha renacido parcialmente debido a innovaciones en econometría. \textcite{froot1994perspectives} muestran que parece haber una convergencia de largo plazo en la paridad de poder de adquisitivo, aunque sería valioso más investigación sobre la cuestión de sesgo de supervivencia.\\
	
	La ecuación fundamental del modelo monetario es una función convencional de demanda de dinero. Utilizando la función correspondiente para el país doméstico y extranjero, y combinando las ecuaciones con la paridad de interés descubierta y la PPP, se obtiene una expresión para el tipo de cambio nominal, que se muestra en la ecuación \ref{flma}. Una derivación completa del modelo se puede consultar en \textcite[86]{frankel1993exchange}, 
	
	\begin{equation}
		s = (m - m^*) - \phi(y-y^*) + \lambda(\pi - \pi^*)
		\label{flma}
	\end{equation}
	
	Donde $s$ es el logaritmo del tipo de cambio \textit{spot}, $m$ es logaritmo de la oferta monetaria, $y$ el logaritmo del ingreso real, $\pi$ la inflación esperada, $\phi$ y $\lambda$ representan elasticidades de demanda de dinero, y el símbolo $^*$ representa las variables extranjeras. La ecuación \ref{flma} indica que el tipo de cambio nominal está determinado por la oferta y demanda por dinero. Por lo tanto, un incremento en la oferta doméstica de dinero provoca una depreciación proporcional \parencite{frankel1993exchange}.
	
	
	\subsubsection{El modelo monetario de precios rígidos}
	Debido a que en el corto plazo pueden aparecer grandes desviaciones de la PPP. La existencia de contratos, información imperfecta, e inercia en los hábitos de consumo hace que los precios no se ajusten instantáneamente, sino gradualmente en el tiempo \parencite{frankel1993exchange}.\\
	
	El modelo de precios rígidos (SPMM, por sus siglas en inglés) asume que la PPP se viola en el corto plazo, pero que se mantiene en el largo plazo \parencite{exchecon}. En este modelo los cambios en la oferta nominal de dinero también representan cambios en la oferta real porque los precios son rígidos, y por lo tanto tienen efectos reales, especialmente en el tipo de cambio \parencite{frankel1993exchange}.\\
	% Acá leí un documento de que talvez ni en el largo plazo se mantiene
	
	El modelo monetario de precios rígidos empieza con el análisis del modelo de Mundell-Fleming (MF). El modelo MF asume que las expectativas sobre el tipo de cambio son estáticas y que hay movilidad perfecta de capitales \parencite{exchecon}. Como las expectativas son estáticas, la paridad descubierta del tipo de interés se vuelve una igualdad entre el tipo de interés doméstico y extranjero \parencite{frankel1993exchange}.\\
	
	En el corto plazo, como los precios son rígidos, de acuerdo con los efectos de liquidez del modelo MF: la tasa de interés cae y genera una salida de capitales, que provoca que la moneda se deprecie instantáneamente más de lo que se depreciaría en el largo plazo. La depreciación ocurre hasta que el tipo de cambio de apreciación futura esperada racionalmente cancele el diferencial de interés. A este fenómeno se describe como ``sobrerreacción'', y se utiliza para distinguir al modelo monetario de precios flexibles \parencite{frankel1993exchange}.\\
	
	Utilizando la derivación del modelo de precios rígidos presentada por \textcite[89]{frankel1993exchange} se obtiene una ecuación monetaria general para la determinación del tipo de cambio: 
	
	\begin{equation}
		s = (m - m^*) - \phi (y-y^*) + \lambda (\pi - \pi^*) - (1/\theta)[(i-\pi) - (i^* - \pi^*)]
		\label{spma}
	\end{equation}
	
	Donde $\theta$ representa la velocidad de ajuste del tipo de cambio, $i$ el tipo de interés nominal, y nuevamente se utiliza el símbolo $^*$ para representar las variables extranjeras. La ecuación \ref{spma} combina la senda de equilibrio monetario de largo plazo con el efecto de sobrerreacción de corto plazo, y añade como variable explicatoria el diferencial del tipo de interés.
	
	\subsection{Evolución histórica del tipo de cambio}
	% Qué es lo que ha pasado con el tipo de cambio en el mundo
	
	La mayoría de las grandes economías industrializadas hicieron flotante su tipo de cambio a principios de 1973, después del sistema de tipos de cambio fijos acordado en la conferencia de Bretton Woods, al finalizar la Segunda Guerra Mundial. Desde entonces, han habido extensas disputas académicas sobre los méritos relativos de los tipos de cambio fijos y flotantes, y esta discusión ha sido llevada a cabo a un nivel hipótetico en su mayoría \parencite{frankelrose1994survey}.\\
	
	El regimen flotante generalizado proporcionó a los economistas datos empíricos para resolver las disputas académicas, así como plantear asuntos de política inmediata. Mucha de la literatura en finanzas internacionales producida en la década después del cambio al régimen flotante se enfocó en el desarrollo y estimación de modelos empíricos para los tipos de cambio \parencite{frankelrose1994survey}.\\
	
	% Regimen de tipo de cambio de Guatemala y aspectos de intervención del bancos
	
	%En Guatemala...\parencite{imfregime}
	
	En Guatemala, el tipo de cambio de cambio más relevante es con el dólar estadounidense, que está bajo un régimen flexible y responde a las fluctuaciones de la oferta y demanda de divisas, congruente con el esquema de metas explícitas de inflación (EMEI), implementado a partir de 2005, que busca anclar las expectativas inflacionarias y mantener un tipo de cambio flexible que permita absorber los choques externos. Asimismo, se encuentra bajo un sistema de flotación administrada, en el cual las autoridades limitan las fluctuaciones del tipo de cambio a corto plazo a través de la intervención en el mercado cambiario y ajustes en la política monetaria. En este esquema no existe un compromiso de política implícito o explícito del tipo de cambio \parencite{banguatregime}.\\
	
	De acuerdo con el informe de política monetaria del \textcite{banguatpolmonet}: ``el Banco de Guatemala, mediante una regla transparente y plenamente conocida por el mercado, participa en el Mercado Institucional de Divisas únicamente con el objetivo de moderar la volatilidad del tipo de cambio nominal del quetzal respecto al dólar estadounidense, sin alterar su tendencia''.

	
	\subsection{Modelos empíricos del tipo de cambio}
	% esto es de modelos empíricos
	A inicios de los años ochenta, algunos de los aparentes éxitos empíricos en la literatura se vieron anulados, y los hallazgos clave se empezaron a tornar negativos, una perspectiva que continúa hasta el presente. El resultado negativo más profundo fue el producido por \textcite{meese1983empirical} que compararon las habilidades de una variedad de modelos de tipo de cambio. El resultado clave fue que ningún modelo estructural pudo mejorar la predicción fuera de la muestra de un modelo de caminata aleatoria en el corto y mediano plazo \parencite{frankelrose1994survey}.\\
	
	Desde la publicación de \textcite{meese1983empirical}, la habilidad de vencer a un modelo de caminata aleatoria se ha vuelto una prueba decisiva de qué tan exitoso es un modelo de tipo de cambio. Se ha vuelto el equivalente de una medida de ajuste $R^2$ por la que se juzga un modelo de tipo de cambio. La razón por la que este hallazgo se ha interpretado como una crítica a los modelos basados en fundamentos macroeconómicos es porque Meese y Rogoff dieron una ventaja injusta a sus modelos, utilizando los datos reales observados para el pronóstico de tipo de cambio \parencite{exchecon}.
	
	
	\subsection{No linealidad}
	
	El modelo monetario es intuitivamente atractivo pero explica muy poco las variaciones en el tipo de cambio. Una explicación es que el tipo de cambio es relativamente insensible a los fundamentos monetarios cerca de los valores de equilibrio del modelo, pero tiende a revertirse fuertemente a los fundamentos cuando la desviación es grande \parencite{neely2002well}.\\
	
	De acuerdo con Heckscher (1916), debido a que existen costos de transacción en el arbitraje internacional, y por lo tanto, la velocidad de ajuste de las desviaciones respecto de la paridad de la PPP no es uniforme como en el marco lineal. \parencite{taylor2004pppdebate}.\\
	
	Según Gradojevic y Yang (2000), el tipo de cambio depende en gran medida de forma no lineal, y según Baillie y McMahon (1989) el tipo de cambio no puede ser pronosticado linealmente. Finalmente, como indican Pippenger y Goering (1998), desde el punto de vista del tipo de cambio como el precio de activo, es probable que contenga no linealidades significativas, así como otros datos de series de tiempo financieras. \parencite[Citado en ][9]{sunythesis}.
