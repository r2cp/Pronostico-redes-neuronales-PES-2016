% Configuración de formato para tesina BANGUAT 2016

% Márgenes
\usepackage{geometry}
\geometry{letterpaper, margin=1in, left=3cm}

% Numeración a los ambientes \paragraph y profundidad del índice
\setcounter{secnumdepth}{4}
\setcounter{tocdepth}{3}

% Formato de las secciones y subsecciones
%\usepackage{titlesec}
%\titleformat*{\section}{\large\bfseries}
%\titleformat*{\subsection}{\normalsize\bfseries}
%\titleformat{\section}{\normalfont\scshape}{\thesection}{1em}{}
%\titleformat{\section}[block]{\large\bfseries}{\thesection.}{0.5cm}{}

% Letra arial
\usepackage{helvet}
\renewcommand{\familydefault}{\sfdefault}

% Indentado
\setlength\parindent{1cm}

% Interlineado 1.5
\renewcommand{\baselinestretch}{1.5}

% Fuente para la mate
\usepackage[osf,sc]{mathpazo}
\usepackage{eulervm}

% Fuente para las gráficas y tablas
\usepackage{caption}
\captionsetup{
	labelfont=bf,
	textfont={normalsize},
	justification=centering}

% Líneas para las tablas
\usepackage{multirow}
\usepackage{booktabs}
% Para poner el largo de la tabla
\usepackage{tabularx}

% Configuración de listings
% Paquetes necesarios
\usepackage{listingsutf8}
\usepackage{color}

% Redefinición de titulos de caption
\renewcommand\lstlistingname{Código}
\renewcommand\lstlistlistingname{Índice de códigos}

% Definición de colores para R
\definecolor{frame_numbers_color}{rgb}{0.5,0.5,0.5}

\definecolor{r_comment_color}{RGB}{221, 0, 221}
\definecolor{r_string_color}{RGB}{0, 170, 0}
\definecolor{r_keyword_color}{RGB}{255, 119, 0}
\definecolor{r_identifier_color}{RGB}{0,0,0}

% Definicion de estilo para R
\lstdefinestyle{customR}{
	backgroundcolor=\color{white},   % choose the background color;
	%basicstyle=\footnotesize,        % the size of the fonts that are used for the code
	basicstyle=\small\ttfamily,
	breakatwhitespace=false,         % sets if automatic breaks should only happen at whitespace
	breaklines=true,                 % sets automatic line breaking
	captionpos=t,                    % sets the caption-position to bottom
	commentstyle=\color{r_comment_color},    % comment style
	deletekeywords={...},            % if you want to delete keywords from the given language
	escapeinside={\%*}{*)},          % if you want to add LaTeX within your code
	extendedchars=true,              % lets you use non-ASCII characters; for 8-bits encodings only, does not work with UTF-8
	frame=single,                    % adds a frame around the code
	framesep=3pt,
	identifierstyle=\color{r_identifier_color},
	keepspaces=true,                 % keeps spaces in text, useful for keeping indentation of code (possibly needs columns=flexible)
	keywordstyle=\mdseries\color{r_keyword_color},       % keyword style
	language=R,                 % the language of the code
	morekeywords={},            % if you want to add more keywords to the set
	numbers=left,                    % where to put the line-numbers; possible values are (none, left, right)
	numbersep=7pt,                   % how far the line-numbers are from the code
	numberstyle=\tiny\color{frame_numbers_color}, % the style that is used for the line-numbers
	rulecolor=\color{black},         % if not set, the frame-color may be changed on line-breaks within not-black text (e.g. comments (green here))
	showspaces=false,                % show spaces everywhere adding particular underscores; it overrides 'showstringspaces'
	showstringspaces=false,          % underline spaces within strings only
	showtabs=false,                  % show tabs within strings adding particular underscores
	stepnumber=1,                    % the step between two line-numbers. If it's 1, each line will be numbered
	stringstyle=\color{r_string_color},     % string literal style
	tabsize=4,                       % sets default tabsize to 2 spaces
	title=\lstname                   % show the filename of files included with \lstinputlisting; also try caption instead of title	
}
\lstset{style=customR, frame=none}


% Para la bibliografía
%\setlength\bibitemsep{1.5\itemsep}
\setlength\bibitemsep{\baselineskip}

% Doble enter después de párrafos
%\nonfrenchspacing

% Para evitar que deje secciones al final de la página
\begin{comment}
\usepackage{etex}
\usepackage{etoolbox}
\makeatletter
\patchcmd{\@afterheading}%
{\clubpenalty \@M}{\clubpenalties 3 \@M \@M 0}{}{}
\patchcmd{\@afterheading}%
{\clubpenalty \@clubpenalty}{\clubpenalties 2 \@clubpenalty 0}{}{}
\makeatother

% Otra posible solución con needspace:

\usepackage{titlesec}
\usepackage{needspace}
...
\titleformat{\section}
{\needspace{1in}\Large\bfseries}{\thesection}{1em}{}

\end{comment}


