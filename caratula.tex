\thispagestyle{empty}
\noindent
Investigación tutelada\\
Maestría en Economía y Finanzas Cuantitativas\\
Profesor Asesor Dr. Guillermo Díaz Castellanos\\
Director\\
Departamento de Economía Empresarial

\vspace{1cm}

\begin{center}
	De acuerdo al dictamen rendido por el\\
	Doctor Guillermo Díaz Castellanos\\
	Asesor del trabajo de investigación
\end{center}

\vspace{1cm}

\begin{center}
	MODELO DE PRONÓSTICO DEL TIPO DE CAMBIO PARA GUATEMALA UTILIZANDO REDES NEURONALES ARTIFICIALES
\end{center}

\vspace{1cm}

\begin{center}
	Presentado por\\
	Rodrigo Rafael Chang Papa
\end{center}

\vspace{0.5cm}

\begin{center}
	En cumplimiento de los requisitos del curso\\
	Seminario de investigación\\
	De la\\
	Maestría en Economía y Finanzas Cuantitativas\\
	Universidad Rafael Landívar\\
\end{center}

\vspace{1cm}

\begin{center}
	Guatemala de la Asunción\\
	Julio de 2016
\end{center}

